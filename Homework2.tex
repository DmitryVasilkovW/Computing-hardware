\documentclass{article}
\usepackage{graphicx}
\usepackage{tikz}
\usepackage[russian]{babel}

\title{ЭВМ}
\author{Выполнил: \textbf{Васильков Дмитрий M3115}}

\begin{document}

\maketitle
\begin{tabular}{|c|c|c|c|}
    \hline
    Адрес & Код команды & Мнемоника & Комментарий \\ \hline
    001 & 0000 & ISZ 000& Необходимо для выхода из подпрограммы. \\ \hline
    002 & F200 & CLA & Очистка аккумулятора. \\ \hline
    003 & F800 & INC & Увеличение регистра А на 1. \\ \hline
    004 & 4008 & ADD 008 & Добавление значения из ячейки 008 в регистр А. \\ \hline
    005 & 3008 & MOV 008 & Присваивает значению ячейки 008 значение из регистра А. \\ \hline
    006 & C801 & BR(001) & Присваивает СК значение по адресу из ячейки 001. \\ \hline
    007 & 0000 & ISZ 000 & Не используется. \\ \hline
    008 & 0000 & ISZ 000 & Нужна для записи результата. \\ \hline
    009 & 0000 & ISZ 000 & Не используется. \\ \hline
    00A & 0015 & ISZ 015& Значение для цикла. \\ \hline
    00B & 0000 & ISZ 000 & Не используется. \\ \hline
    00C & F200 & CLA & Очистка аккумулятора. \\ \hline
    00D & 480A & ADD(00А) & Добавление в А значения по адресу из 00A, 00A++. \\ \hline
    00E & 9010 & BPL 010 & Проверка: если в регистре А >= 0, то СК == 010. \\ \hline
    00F & 2001 & JSR 001 & Необходимо для работы подпрограммы. \\ \hline
    010 & 0013 & ISZ 013 & Нужна для подпрограммы. \\ \hline
    011 & C00C & BR 00С& Присваивает регистру СК значение 00С. \\ \hline
    012 & F000 & HLT & Останавливает ЭВМ. \\ \hline
    013 & FFFB & HZF & Число для цикла. \\ \hline
    014 & 0000 & ISZ 000& Не используется. \\ \hline
    015 & 0000 & ISZ 000& Значение для вычислений. \\ \hline
    016 & CCCE & BR(4СЕ) & Значение для вычислений. \\ \hline
    017 & 90BA & BPL 0ВА& Значение для вычислений. \\ \hline
    018 & 0000 & ISZ 000& Значение для вычислений. \\ \hline
    019 & EEBB & IN(00В) & Значение для вычислений. \\ \hline
\end{tabular}

\begin{center}
    \hline Описание программы

Программа считает количество чисел, которые строго меньше 0.
Область представления данных и результата: 015 – 019 и 008.
Начало программы – 00C.
Конец программы – 012.

Вывод

Данная работа помогла научиться писать комплекс программ, состоящий из программы
и подпрограммы и обеспечивающий подсчет количества требуемых элементов массива
данных.
\end{center}
\end{document}
