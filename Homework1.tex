\documentclass{article}
\usepackage{graphicx}
\usepackage{tikz}
\usepackage[russian]{babel}

\title{ЭВМ}
\author{Выполнил: \textbf{Васильков Дмитрий M3115}}
\date{16 Марта 2023}

\begin{document}

\maketitle


Домашнее задание № 1

Выполнение арифметических операций с двоичными числами.

Цель задания - овладеть простейшими навыками перевода чисел в различные системы счисления и выявить ошибки, возникающие из-за их ограниченной разрядности.

1) По заданному варианту исходных данных получить набор десятичных чисел: $X_1=A, X_2=C, X_3=A+C, X_4=A+C+C, X_5=C-A, X_6=65536-X_4, X_7=-X_1, X_8=-X_2, X_9=-X_3, X_{10}=-X_4, X_{11}=-X_5, X_{12}=-X_6.$

Выполнить перевод десятичных чисел $X_1,...,X_{12}$ в двоичную систему счисления, получив их двоичные эквиваленты $B_1,...,B_{12}$ соответственно. Для представления двоичных чисел $B_1,...,B_{12}$ использовать 16-разрядный двоичный формат со знаком.

Для контроля правильности перевода выполнить обратный перевод двоичных чисел в десятичные и подробно проиллюстрировать последовательность прямого и обратного перевода для чисел $X_1, B_1, X_7$ и $B_7$.

2) Выполнить следующие сложения двоичных чисел: $B_1+B_2, B_2+B_3, B_7+B_8, B_8+B_9, B_2+B_7, B_1+B_8.$ Для представления слагаемых и результатов сложения использовать 16-разрядный двоичный формат со знаком.

Результаты сложения перевести в десятичную систему счисления, сравнить с соответствующими десятичными числами. Дать подробные комментарии полученным результатам.

Вариант 14:

Операнд $A = 1978$

Операнд $C = 15516$

Решение:

Первая часть:

$X_1 = A = 1978$

$X_2 = C = 15516$

$X_3 = A + C = 17494$

$X_4 = A + C + C = 33010$

$X_5 = C - A = 13538$

$X_6 = 65536 - X_4 = 32526$

$X_7 = -X_1 = -1978$

$X_8 = -X_2 = -15516$

$X_9 = -X_3 = -17494$

$X_{10} = -X_4 = -33010$

$X_{11} = -X_5 = -13538$

$X_{12} = -X_6 = -32526$

$B_1=0000\:0111\:1011\:1010 = 0\cdot 2^{15} + 0\cdot 2^{14} + 0\cdot 2^{13} + 0\cdot 2^{12} + 0\cdot 2^{11} + 1\cdot 2^{10} + 1\cdot 2^9 + 1\cdot 2^8 + 1\cdot 2^7 + 0\cdot 2^6 + 1\cdot 2^5 + 1\cdot 2^4 + 1\cdot 2^3 + 0\cdot 2^2 + 1\cdot 2^1 + 0\cdot 2^0 = 1978$

$B_2=0011\:1100\:1001\:1100 = 15516$

$B_3=0100\:0100\:0101\:0110 = 17494$

$B_4=1000\:0000\:1111\:0010 = -242$

$B_5=0011\:0100\:1110\:0010 = 13538$

$B_6=0111\:1111\:0000\:1110 = 32526$

$B_7=1111\:1000\:0100\:0110$ доп (0000 0111 1011 1010 прямой $\rightarrow$ 1111 1000 0100 0101 обратный) = $-30790$

$B_8=1100\:0011\:0110\:0100$ доп (0011 1100 1001 1100 прямой $\rightarrow$ 1100 0011 0110 0011 обратный) = $-17252$

$B_9=1011\:1011\:1010\:1010$ доп (0100 0100 0101 0110 прямой $\rightarrow$ 1011 1011 1010 1001 обратный) = $-15274$

$B_{10}=1111\:1111\:0000\:1110$ доп (1000 0000 1111 0010 прямой $\rightarrow$ 0111 1111 0000 1101 обратный) = $-32526$

$B_{11}=1100\:1011\:0001\:1110$ доп (0011 0100 1110 0010 прямой $\rightarrow$ 1100 1011 0001 1101 обратный) = $-19230$

$B_{12}=1000\:0000\:1111\:0010$ доп (0111 1111 0000 1110 прямой $\rightarrow$ 1000 0000 1111 0001 обратный) = $-242$

Обратный перевод:

$B_1=0000\:0111\:1011\:1010 = 0\cdot 2^{15} + 0\cdot 2^{14} + 0\cdot 2^{13} + 0\cdot 2^{12} + 0\cdot 2^{11} + 1\cdot 2^{10} + 1\cdot 2^9 + 1\cdot 2^8 + 1\cdot 2^7 + 0\cdot 2^6 + 1\cdot 2^5 + 1\cdot 2^4 + 1\cdot 2^3 + 0\cdot 2^2 + 1\cdot 2^1 + 0\cdot 2^0 = 1024 + 512 + 256 + 128 + 32 + 16 + 8 + 2 = 1978$

$B_7=1111\:1000\:0100\:0110$ в доп. коде $\rightarrow$ 1111 1000 0100 0101 обратный код $\rightarrow$ 0000 0111 1011 1010 (2) $\rightarrow$ 1978(10) значит 1111 1000 0100 0110 $= -1978$

Вторая часть:

$B_1 + B_2 = 0100\:0100\:0101\:0110 = 17494$ $(17494)$

$0000\:0111\:1011\:1010 + 0011\:1100\:1001\:1100 \rightarrow 0100\:0100\:0101\:0110$

$B_2 + B_3 = 1000\:0000\:1111\:0010 = -242$ тк переполнение $(33010)$

$0011\:1100\:1001\:1100 + 0100\:0100\:0101\:0110 \rightarrow 1000\:0000\:1111\:0010$

$B_7 + B_8 = 1011\:1011\:1010\:1010 = -17494$ $(-17494)$

Обратный 1011 1011 1010 1001

Прямой 0100 0100 0101 0110

$1111\:1000\:0100\:0110 + 1100\:0011\:0110\:0100 \rightarrow 1011\:1011\:1010\:1010$

$B_8 + B_9 = 0111\:1111\:0000\:1110 = 242,$ тк переполнение $(-33010)$
$1100\:0011\:0110\:0100 + 1011\:1011\:1010\:1010 \rightarrow 0111\:1111\:0000\:1110$

$B_2 + B_7 = 0100\:0000\:0001\:0010 = -15210$ $(-15210)$

$0011\:1100\:1001\:1100 + 1111\:1000\:0100\:0110 \rightarrow 0100\:0000\:0001\:0010$

$B_1 + B_8 = 1100\:0010\:0001\:1100 = -1362$ $(-1362)$

$0000\:0111\:1011\:1010 + 1100\:0011\:0110\:0100 \rightarrow 1100\:1011\:0001\:1110$

В результате выполнения задания были получены двоичные эквиваленты десятичных чисел, выполнялись операции сложения в двоичной системе со знаком, а также выполнялись переводы между двоичной и десятичной системами счисления, что позволило убедиться в правильности полученных результатов. Также были выявлены ошибки, связанные с ограниченной разрядностью чисел и переполнениями.

\end{document}
