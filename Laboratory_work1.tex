\documentclass{article}
\usepackage{graphicx}
\usepackage{tikz}
\usepackage[russian]{babel}

\title{ЭВМ}
\author{Выполнил: \textbf{Васильков Дмитрий M3115}}

\begin{document}

\maketitle
\begin{table}[h]
\centering
\begin{tabular}{|c|c|c|c|}
\hline
Адрес & Код команды & Мнемоника & Комментарий \\
\hline
017 & 0000 & ISZ 000& Не используется. \\
018 & F1AA & NOP & Значение для вычислений. \\
019 & 7C89 & HZA7 (489) & Значение для вычислений. \\
01A & 2A5A & JSR (25A) & Значение для вычислений. \\
01B & 0000 & ISZ 000 & Значение для вычислений. \\
01C & F200 & CLA & Очистка аккумулятора. \\
01D & 4018 & ADD 018 & A += 018. \\
01E & 501A & ADC 01A & 01А + А + 1, если C равен 1. \\
01F & 301B & MOV 01B& Присваивает ячейке по адресу 01В значение из регистра А. \\
020 & F200 & CLA & Очистка аккумулятора. \\
021 & 4019 & ADD  019 & Присваивает регистру А = А + 019. \\
022 & 101B & AND 01B & Присваивает регистру А = А + 01B. \\
023 & 301B & MOV 01B & Присваивает ячейке по адресу 01В значение из А. \\
024 & F000 & HLT & Выключает ЭВМ. \\
\hline
\end{tabular}
\end{table}

\begin{center}
    1) Формула: 01B = 019 & (018 + 01А + С).\\
2) Назначение программы – запись в ячейку 01В результат побитового И
между значением в ячейке 019 и результатом суммы 018, 01А и С.\\
3) Область представления данных и результатов: 018 – 01В. 4)
Расположение в памяти ЭВМ программы, исходных данных и
результатов: 018 – 01B.\\
5) Адреса первой и последней выполняемой команд программы: 01С и
024.
\end{center}



\end{document}
